\section{Intro}
\label{sec:intro}
\textit{"You are going to develop a travel-planning system in which you will need to implement a method for computing the cheapest route between destinations. \\
Data about the destinations and possible routes between them are placed in a file (to be found on black board next to the assignment) where each line contains a destination followed by the cities to which you can travel and the associated cost. \\
Notice that even though there is a route from A to B, there might not be one from B to A."}


\section{Solution}
\subsection{Question \#1}
\todo[inline]{Some kind of pseudo code}
A routine for loading in the file and appropriate data structures for representing the data is shown in appendices \ref{app:FileHandle} and \ref{app:Vertex}.\\
We are using a Hash-map where from cities are associated with accesable cities and thier egde cost.



\subsection{Question \#2}
\todo[inline]{Some kind of pseudo code}
\todo[inline]{Print from Odense for later use}

As mentioned before the from cities is associate to cities with a giving cost. The apporach for




\begin{lstlisting}
void Graph::printFrom(std::string from){
	if(vertices.find(from) == vertices.end()){
		std::cout <<  "City \"" + from +  "\" not found" << std::endl;
		return ;
	}
	for(auto it = vertices[from]->edge.get_container().begin() ; it != vertices[from]->edge.get_container().end(); ++it ){
		std::cout << "To: " << it->first->element << " Cost: " <<it->second  << std::endl;
	}
} 
\end{lstlisting}
\bigskip

Console output is showed below:
\begin{lstlisting}


CONSOLE OUTPUT

\end{lstlisting}


\subsection{Question \#3}
We have chosen to use the Dijkstras algorithm for computing the quickest route between two destinations. The properties of Dijkstras algorithm producing a shortest path for a graph with non-negative edge path costs. \\
The Dijkstras creates the graph of vetrices and egdes from a priority queues which ensure the algorithm at alltime  get next vertex with the lowest egde cost. It was nessaray to make an overloadoperator \emph{Comp} (appendix \ref{app:vertex_h}) to get the queue match our data structure and sorting correctly for the \emph{cost}.

\todo[inline]{Some kind of pseudo code}
\begin{lstlisting}
psoducode
\end{lstlisting}





\section{Examples and Benchmarks}
\subsection{Ten different from and to cities}
\begin{lstlisting}


CONSOLE OUTPUT

\end{lstlisting}


\subsection{Odense to Aalborg}
\begin{lstlisting}


CONSOLE OUTPUT

\end{lstlisting}

	
\subsection{Odense to Holstebro}

\begin{lstlisting}


CONSOLE OUTPUT

\end{lstlisting}

\subsection{Odense to Humlebæk}

\begin{lstlisting}


CONSOLE OUTPUT

\end{lstlisting}



\section{Conclusion}


