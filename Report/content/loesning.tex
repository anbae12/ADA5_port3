\section{Intro}
\label{sec:intro}
\textit{"You are going to develop a travel-planning system in which you will need to implement a method for computing the cheapest route between destinations. \\
Data about the destinations and possible routes between them are placed in a file (to be found on black board next to the assignment) where each line contains a destination followed by the cities to which you can travel and the associated cost. \\
Notice that even though there is a route from A to B, there might not be one from B to A."}


\section{Solution}
\subsection{Question \#1}
A routine for loading in the file and appropriate data structures for representing the data is shown in appendices \ref{app:FileHandle} and \ref{app:Vertex}.\\
We are using a Hash-map where from cities are associated with accesable cities and thier egde cost.

\subsection{Question \#2}
As mentioned before the from cities is associate to cities with a giving cost.  The apporach for printing the "to cities" is showed by the psodocode and the listing below. 

\begin{lstlisting}
void Graph::printFrom(std::string from){
	if(vertices.find(from) == vertices.end()){
		std::cout <<  "City \"" + from +  "\" not found" << std::endl;
		return ;
	}
	for(auto it = vertices[from]->edge.get_container().begin() ; it != vertices[from]->edge.get_container().end(); ++it ){
		std::cout << "To: " << it->first->element << " Cost: " <<it->second  << std::endl;
	}
} 
\end{lstlisting}
\bigskip

Accesable cities from "Odense" is copyed from the console output and showed in table \ref{tb:fromodense}. 
\begin{figure}[th!]
\centering
\begin{tabular}{l|l}
To &Cost\\\hline
Stubbekøbing & 20\\
Værløse & 22\\
Hjørring & 33\\
København & 29\\
Søllested & 54\\
Gedved & 62\\
Broby & 67\\
Odder & 48\\
Hørning & 34\\
Spenstrup & 144\\
Dronningmølle & 73\\
Karup & 204\\
Kalundborg & 173\\
Kerteminde & 193\\
Jerup & 87\\
Hovborg & 221\\
Vedbæk & 163\\
Rønde & 187\\
Mørkøv & 47\\
Langebæk & 234\\
Langeskov & 191\\
Ålsgårde & 177\\
Nysted & 102\\
\end{tabular}
\captionsetup{type=table}
\caption[tekst i indholdsfortegnelsen]{Associated arrival cities and cost from Odense.}
\label{tb:fromodense}
\end{figure}

\subsection{Question \#3}
We have chosen to use the Dijkstras algorithm for computing the quickest route between two destinations. The properties of Dijkstras algorithm producing a shortest path for a graph with non-negative edge path costs. \\
The Dijkstras creates the graph of vetrices and egdes from a priority queues which ensure the algorithm at alltime  get next vertex with the lowest egde cost. It was nessaray to make an overloadoperator \emph{Comp} (appendix \ref{app:vertex_h}) to get the queue match our data structure and sorting correctly for the \emph{cost}.\\

\begin{lstlisting}
DijkResult Dijkstras::Run(std::string from, std::string to){
	if (mGraph->vertices.find(from) == mGraph->vertices.end()) {
		std::cout<<"Not found: "<<from<<std::endl;
		exit(0);
	}
	if (mGraph->vertices.find(to) == mGraph->vertices.end()) {
		std::cout<<"Not found: "<<to<<std::endl;
		exit(0);
	}
	std::string depTown = from;
	std::string arTown = to;

	mGraph->vertices[from]->dist=0;
	dijkstrasQueue.push(mGraph->vertices[from]);
	while (!dijkstrasQueue.empty()) {
		from = dijkstrasQueue.top()->element;
		dijkstrasQueue.pop();
		while (!mGraph->vertices[from]->edge.empty()) {
			std::string to = mGraph->vertices[from]->edge.top().first->element;
			int cost  = mGraph->vertices[from]->edge.top().second;

			int edgeplusnode = cost + mGraph->vertices[from]->dist;

			if ( edgeplusnode <  mGraph->vertices[to]->dist) {
				mGraph->vertices[to]->dist=edgeplusnode;
				mGraph->vertices[to]->from=mGraph->vertices[from];
			}
			dijkstrasQueue.push(mGraph->vertices[from]->edge.top().first );
			mGraph->vertices[from]->edge.pop();
		}
	}
	auto route = path(mGraph->vertices[depTown], mGraph->vertices[arTown]);
	return DijkResult(route.second,mGraph->vertices[arTown]->dist,0, route.first);
}
\end{lstlisting}

\section{Examples and Benchmarks}
\subsection{Ten different from and to cities}
\begin{lstlisting}


CONSOLE OUTPUT

\end{lstlisting}


\subsection{Odense to Aalborg}
\begin{lstlisting}


CONSOLE OUTPUT

\end{lstlisting}

	
\subsection{Odense to Holstebro}

\begin{lstlisting}


CONSOLE OUTPUT

\end{lstlisting}

\subsection{Odense to Humlebæk}

\begin{lstlisting}


CONSOLE OUTPUT

\end{lstlisting}



\section{Conclusion}


