\section{Intro}
\label{sec:intro}
\textit{"You are going to develop a travel-planning system in which you will need to implement a method for computing the cheapest route between destinations. \\
Data about the destinations and possible routes between them are placed in a file (to be found on black board next to the assignment) where each line contains a destination followed by the cities to which you can travel and the associated cost. \\
Notice that even though there is a route from A to B, there might not be one from B to A."}


\section{Solution}
\subsection{Question \#1}
A routine for loading in the file and a appropriate data structure for representing the data is shown in appendices  \ref{app:Vertex} and \ref{app:FileHandle}. We are using a Hash-map where from-cities are associated with accesable to-cities and their edge cost.\\
The reason why the group use a hash-map is to make the lookup time constant. \\
Each vertex is represented by an instance of the class called Vertex. This class includes a string with the name of the vertex, a priority queue including all of its edges. Each edge consists of a std::pair, where the first value is a pointer to an adjacent vertex, and the second value is the cost. 
The reason why a priority queue has been chosen, instead of an ordinary vector, is to avoid using a sorting algorithm to sort the vector.\\
If the vector was chosen the complexity would increase to \(O(|V|^2)\) insted of \(O(|E| + |V|log(|V|))\). \\
To parse the file, the following approach has been used:
\bigskip
\begin{lstlisting}
foreach(line in file){
	from = getFrom(line)
	while(getline(line,to,',') && getline(line,cost,',')) {
		graph->addVertex(to)
		graph->addEdge(from, to, cost)
	}	
}
\end{lstlisting}
\bigskip
For each line in the file, the from-city is extracted. For the rest of the line, the while loop will parse to and cost. If getline can not get a value, it will return false. For each to-city and associated cost, it will create a vertex and add an edge between the two cites. If the from-city already exist it will throw an exception. This has been omitted in the code above for simplicity.


\subsection{Question \#2}
As mentioned before the from-cities are associate to-cities with a giving cost. The approach for printing the to-cities is showed by the pseudocode and the listing below. 
\bigskip
\begin{lstlisting}
printFrom(from-city){
	check all vertices 
	if ( end of vertices ) return "from-city not found"
	foreach( adjacent vertex to from-city){
	    print associated city and giving cost
    }
}
\end{lstlisting}
The iteration through all adjacent vertices has been implemented by inheriting the priority queue. The underlaying container is protected and there by accessable through inheriting and implementing a get\_container() method in the derived class that returns the vector.
\bigskip
It first checks if the from-city is in the hash-map. If not, it returns and will not print any associated cities. If found, it will iterate through all the associated cities and their cost.\\
Accessible cities from "Odense" is copied from the console output and showed in table \ref{tb:fromodense}. 
\begin{figure}[th!]
\centering
\begin{tabular}{l|l}
To &Cost\\\hline
Stubbekøbing & 20\\
Værløse & 22\\
Hjørring & 33\\
København & 29\\
Søllested & 54\\
Gedved & 62\\
Broby & 67\\
Odder & 48\\
Hørning & 34\\
Spenstrup & 144\\
Dronningmølle & 73\\
Karup & 204\\
Kalundborg & 173\\
Kerteminde & 193\\
Jerup & 87\\
Hovborg & 221\\
Vedbæk & 163\\
Rønde & 187\\
Mørkøv & 47\\
Langebæk & 234\\
Langeskov & 191\\
Ålsgårde & 177\\
Nysted & 102\\
\end{tabular}
\captionsetup{type=table}
\caption[tekst i indholdsfortegnelsen]{Associated arrival cities and cost from Odense.}
\label{tb:fromodense}
\end{figure}

\subsection{Question \#3}
We have chosen to use the Dijkstra's algorithm for computing the distances to each vertex from a given vertex.
\bigskip
\begin{lstlisting}
DijkResult Run(from, to){
	if (from from in list) {
		cout<<"Not found: " << to << endl
		exit(0)
	}

	fromVertex.dist = 0;

	dijkstrasQueue.add(fromVertex);

	while (dijkstrasQueue not empty) {

		from = dijkstrasQueue.top()
		while(while adjacent list is not empty){

			string to = vertices[from]->edge.top().first->element
			int cost  = vertices[from]->edge.top().second

			int edgeplusnode = cost + vertices[from]->dist

			if ( edgeplusnode <  vertices[to]->dist) {
				vertices[to]->dist = edgeplusnode
				vertices[to]->from = vertices[from]
			}
			dijkstrasQueue.push(vertices[from]->edge.top().first )
			vertices[from]->edge.pop()
		}
	}
	auto route = path(vertices[depTown], vertices[arTown]);
	return DijkResult(route.second, vertices[arTown]->dist, route.first);
}
\end{lstlisting}
\bigskip
The algorithm works by starting at the from-vertex, visit each adjacent vertex (to-cities) and add them to the Dijkstra priority queue. Then add cost and "move" to the node with the lowest cost.
The algorithm will again visit each adjacent vertex and update the cost if it's smaller than the exiting cost.
By default all vertices cost are initialized to the maximum value an integer can represent, hereby an "infinity cost".

\section{Examples and Benchmarks}
\subsection{Ten different from and to cities}
Table \ref{tb:fromtoten} shows the planning duration from different from-cities and to-cities.
\begin{figure}[th!]
\centering
\begin{tabular}{l|l|l}
From-city & To-city & Duration \\\hline
Odense & Aalborg & 60,752 [ms] \\
Næstved & Odense & 62,742 [ms] \\
Balle & Janderup & 61,288 [ms] \\
Beder & Glumsø & 63,465 [ms] \\
Blokhus & Glostrup & 62,664 [ms] \\
Borre & Vadum & 63,448 [ms] \\
Bredebro & Gistrup & 63,218 [ms] \\
Bælum & Hornsyld & 62,492 [ms] \\
Fakse & Bredebro & 61,774 [ms] \\
Farum & Hadsten & 61,869 [ms] \\\hline
Average runtime && 62,37 [ms]
\end{tabular}
\captionsetup{type=table}
\caption[tekst i indholdsfortegnelsen]{Duration for ten different from-cities and to-cities.}
\label{tb:fromtoten}
\end{figure}



\subsection{Test from from-city to to-cities}
Table \ref{tb:fromonetocities} the shows planning duration from one from-city to three different to-cities. As you see in the table \ref{tb:fromonetocities} the first planning duration is fairly high according to the two next route plannings.
\begin{figure}[th!]
\centering
\begin{tabular}{l|l|l}
From-city & To-city & Duration \\\hline
Odense & Næstved & 62,802 [ms] \\
Odense & København & 0,005 [ms] \\
Odense & Vadum & 0,005 [ms]  
\end{tabular}
\captionsetup{type=table}
\caption[tekst i indholdsfortegnelsen]{Duration for planning from Odense to three cities.}
\label{tb:fromonetocities}
\end{figure}


\subsection{Planning, Shifts and Ticket price}
Following three examples show the route planning from a giving city to a giving city. It return the path and cheapest price and the planning duration.

\subsubsection{Odense to Næstved}
\textbf{Departure:} Odense \\
\textbf{Arrival:}   Næstved \\
\textbf{Shifts:}    6: Odense $\rightarrow$ Værløse $\rightarrow$ Rødvig Stevns $\rightarrow$ Humble $\rightarrow$ Skørping $\rightarrow$ Kerteminde $\rightarrow$ Næstved\\
\textbf{Ticket:}    64,- DKK \\
\textbf{Duration:}  62,306 [ms] 
\subsubsection{Odense to Sønderborg}
\textbf{Departure:} Odense \\
\textbf{Arrival: }  Sønderborg \\ 
\textbf{Shifts}:    5: Odense $\rightarrow$ Værløse $\rightarrow$ Hornsyld $\rightarrow$ Ebberup $\rightarrow$ Vig $\rightarrow$ Sønderborg \\
\textbf{Ticket}:    88,- DKK \\
\textbf{Duration:}  62,148 [ms]
\subsubsection{Vadum to Vejle}
\textbf{Departure:} Vadum\\
\textbf{Arrival}:   Vejle\\
\textbf{Shifts:}    7: Vadum $\rightarrow$ Højbjerg $\rightarrow$ Glesborg $\rightarrow$ Gjern $\rightarrow$ Assels Øster $\rightarrow$ Ærøskøbing $\rightarrow$ Børkop $\rightarrow$ Vejle \\
\textbf{Ticket:}    62,- DKK \\
\textbf{Duration:}  60,99 [ms]

\section{Conclusion}
By using the Dijkstra's algorithm and having the data in a suitable data structure it was possible to get a nice cheapest route planning algorithm. To make a suitable algorithm which have an average duration of 62,37 [ms] (table \ref{tb:fromtoten}) for planning the cheapest route between ten different from-cities and to-cities, we conclude the portfolio is successful. 

